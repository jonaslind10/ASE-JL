\section{Lektion 17-04-2018}

\begin{enumerate}
	\item White noise
	\item Shot noise
	\item Støjfaktor
\end{enumerate}

\noindent\fbox{\parbox{\textwidth}{
	\begin{itemize}
	\item \textbf{Pensum:} 
	\begin{itemize}
		\item CB, Ch 8 p. 194-195
		\item CB, Ch 5 p. 115-123 
		\item CB, Ch 6 p. 152-161
		\item Note: Noise-Calculations.pdf
	\end{itemize} 
	\item \textbf{Opgaver:} 
	\begin{itemize}
		\item Lektion 10-2
		\item Ex2010 opg. 2
		\item Ex2012 opg. 2
	\end{itemize}
\end{itemize}
}} \vspace{3mm}

\subsection{White noise (Thermal)}
\begin{itemize}
	\item Noise from each component adds to the	receiver's noise floor.
	\item Sets the limit on the minimum signal level that can be detected.
	\item PSD is power contained within a	given bandwidth (watts per hertz).
	\item Thermal noise is the minimum amount of noise related to temperature.
	\begin{itemize}
		\item $k=1.38 \cdot 10^{-20}\si{\milli\watt/\kelvin}$ Boltzmann's constant.
		\item $T=$ temperature in degrees Kelvin \si{\kelvin}.
		\item $B=$ noise bandwidth in hertz \si{\hertz}.
	\end{itemize}
\end{itemize}

\begin{equation}
kTB = (1.38\cdot 10^{-23}\si{\joule/\kelvin})(293\si{\kelvin})(1\si{\hertz})
\end{equation}

\begin{itemize}
	\item Increase in bandwidth comes increase in noise power.
	\item Can be expressed as RMS voltage.
\end{itemize}

\begin{equation}
V_n = \sqrt{4kTBR_N}
\end{equation}

\subsection{Shot noise (Schottky)}
\begin{itemize}
	\item Particle-like nature of the charge carriers.
	\item Random current change.
\end{itemize}

\begin{equation}
I_n^2=2qI_{dc}B
\end{equation}

\begin{itemize}
	\item $I_n^2 =$ mean sqaure noise current.
	\item $q =1.6\cdot 10^{-19} \,\si{\coulomb}$ electron charge.
	\item $I_{dc} =$ direct current in amperes (\si{\ampere}).
	\item $B = $ bandwidth in hertz (\si{\hertz}).
\end{itemize}

\begin{itemize}
	\item Can be expressed as RMS current.
\end{itemize}

\begin{equation}
I_n=\sqrt{2qI_{dc}B}
\end{equation}

\subsection{Noise Figure}
\begin{itemize}
	\item Components are characterized by several parameters.
	\begin{itemize}
		\item Noise Figure (NF)
		\item Noise Factor (F)
		\begin{itemize}
			\item Ratio of SNR at output compared to SNR at input.
		\end{itemize}
	\end{itemize}
\end{itemize} 

\begin{equation}
Noise \:Factor = \left(\dfrac{Output\:SNR_2}{Input\:SNR_1}\right)
\end{equation}

\begin{equation}
Noise \:Figure = 10\log\left(\dfrac{Output\:SNR_2}{Input\:SNR_1}\right)
\end{equation}

\begin{itemize}
	\item Noise Figure for a passive device is equal to the insertion loss of
	the device.
	\item Calculation of noise contributions of stages in cascade (Friis).
	\begin{itemize}
		\item Noise Factor of the first stage in the system ($F_1$) has a dominant effect on the overall noise performance.
	\end{itemize}
\end{itemize}

\begin{equation}
NF_{sys} = 10\log\left(\dfrac{F_1+(F_2-1)}{A_1}\right)
\end{equation}

\begin{itemize}
	\item $F =$ noise factor, equivalent to $10^{\frac{NF}{10}}$.
	\item $A=$ numerical power gain, equivalent to $10^{\frac{G}{10}}$.
	\item $G$ power gain in \si{\decibel}.
\end{itemize}

\begin{equation}
F_{sys} = F_1+\dfrac{F_2-1}{G_1}+\dfrac{F_3-1}{G_1G_2}+\dfrac{F_n-1}{G_1G_2...G_{n-1}}
\end{equation}

\begin{itemize}
	\item $F =$ noise factor of each stage.
	\item $G=$ numerical power gain in \si{\decibel}.
\end{itemize}