\section{Lektion 08-02-2018}
\subsection{Simulator and JTAG debugging}
\begin{enumerate}
	\item To be able to use the Atmel Studio 7 simulator.
	\item To be able to set up and use the Atmel - ICE for debugging purposes
\end{enumerate}

\subsubsection{Exercise, Part 1}
\begin{itemize}
	\item Create a project and write a C program (intended for debugging).
\end{itemize}

\subsubsection{Exercise, Part 2}
\begin{itemize}
	\item Simulate the program using the Atmel Studio Simulator ("debugger"). The simulator is integrated in the Atmel Studio IDE.
	\item When using the simulator, the program execution can be simulated in Atmel Studio (not in real time). 
	\item \textbf{Important} to disable the compiler optimization.
	\begin{itemize}
		\item Select "Project" $\:\Rightarrow$ "Properties" (Alt + F7).
		\item Select "Toolchain", and mark "Optimization".
		\item Select "Optimization Level" to "None".
	\end{itemize}
	\item  Remember after debugging, normally the "Optimization Level" should be set to	generate effective code.
	\item Select "Debug" $\:\Rightarrow$ "Start Debugging and Break" (ALT + F5).
	\item Note that the program is "just" simulated, it will take more time than executing the program on the microcontroller.
	\item Registers can be watched in the window "Processor".
	\item Register contents that has been changed by an instruction, will be marked with a red color.
	\item "Stop Watch" show the exact time elapsed since program execution started.
	\item Monitoring specific variables and registers can be done using the "Watch Window". 
	\begin{itemize}
		\item Select "Debug" $\:\Rightarrow$ "Windows" $\:\Rightarrow$ "Watch" $\:\Rightarrow$ "Watch1".
	\end{itemize} 
	\item  Monitoring I/O registers of the microcontroller using "I/O View".
	\begin{itemize}
		\item Select "Debug" $\:\Rightarrow$ "Windows" $\:\Rightarrow$ "I/O View".
	\end{itemize}
\end{itemize}

\subsubsection{Exercise, Part 3}
\begin{itemize}
	\item  Use the Atmel-ICE for debugging the program while it is
	executing in target.
	\item Possible since Mega2560 has an on chip hardware JTAG interface. 
	\item Advantage is that the program is executed in real time.
	\item Disadvance is that 4 pins of PORTF are allocated for the JTAG interface.
	\item \textbf{Important} to enable JTAG interface by setting a fuse. Can only be done using ISP programming. 
	\item Connect the ISP 6 pin connector of the ICE to the Mega2560 ISP connector (plastic tab $\rightarrow$ Mega2560 chip).
	\begin{itemize}
		\item Select "Device Programming".
		\item Select Atmel-ICE as tool and interface as	ISP.
	\end{itemize}
	\item The Arduino board comes with a boot loader installed. When using the ICE, we will probably erase the boot loader.
	\item Start by reading the flash content of your board to a .hex file.
	\item Go to tab "Fuses" and check "JTAGEN". Click Program.
	\item Set up Atmel Studio to use the Atmel-ICE for debugging instead of the simulator.
	\begin{itemize}
		\item Select "Project Properties" (Alt + F7).
		\item Select "Tool" $\:\Rightarrow$ "Atmel-ICE" as debugger $\:\Rightarrow$ "JTAG" as interface. 
	\end{itemize}
	\item Connect the mini squid cable of the Atmel-ICE.
	\begin{itemize}
		\item Signal \space\space ICE port pin \space Mega2560 pin
		\item TCK \space\space\space 1 \space\space\space\space\space\space\space\space\space\space\space\space\space\space\space\space\space PORTF, 4
		\item TMS \space\space\space 5 \space\space\space\space\space\space\space\space\space\space\space\space\space\space\space\space\space PORTF, 5
		\item TDO \space\space\space 3 \space\space\space\space\space\space\space\space\space\space\space\space\space\space\space\space\space PORTF, 6
		\item TDI \space\space\space\space 9 \space\space\space\space\space\space\space\space\space\space\space\space\space\space\space\space\space PORTF, 7
	\end{itemize}
	\item \textbf{Restore} the Arduino with the original bootloader, program the device with the .hex file, saved earlier.
	\item To free the JTAG pins of PORT F, you will have to clear the JTAG
	fuse "JTAGEN".
\end{itemize}

